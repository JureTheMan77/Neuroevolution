Dodatek vsebuje rezultate testiranj programa na zgoraj opisanih podatkovnih množicah po načrtu razdelka~\ref{sec:nacrt-poskusov}.
Rezultati so navedeni na naslednji način:
\begin{itemize}
    \item vsak razdelek zajema rezultate testiranja programa na določeni podatkovni množici,
    \item znotraj razdelka je navedena tabela z naborom inicializacijskih parametrov, katere smo uporabili pri testiranju,
    \item sledijo ji dejanski rezultati za vsak nabor parametrov:
    \begin{itemize}
        \item tabela s točnostjo in MCC najboljših agentov vsakega od petih zagonov,
        \item matrika zmot najbolj točnega agenta in agenta z največjim MCC,
        \item graf točnosti in MCC populacije, kateri pripada najboljši agent, skozi generacije,
        \item vizualizacija najbolj točnega agenta in agenta z največjim MCC.
    \end{itemize}
\end{itemize}

Razdelek Random forest (\ref{sec:random-forest-test}) vsebuje rezultate pristopa naključnih gozdov, pridobljenih s programom Orange.
Za vsako podatkovno množico je podana tabela točnosti in MCC glede na število dreves.
Podane so tudi matrike zmot, kjer je smiselno.

\section{Iris}\label{sec:dodatek-iris-test}
%% arrowLength=4
%% linkWidth=5
%% input fy=100*node.pos
%% output fx=400
%% output fy=100*node.pos+50
%% MAX_FONT_SIZE=8
\begin{table}[H]
    \begin{center}
        \begin{tabular}{||l c c c||}
            \hline
            & 1        & 2        & 3 \\ [0.5ex]
            \hline
            velikost populacije               & 100      & 150      & 200      \\
            \hline
            največje število globokih vozlišč & 5        & 10       & 15       \\
            \hline
            največje število povezav          & 10       & 20       & 30       \\
            \hline
            največje število prečkanj         & 2        & 2        & 3        \\
            \hline
            delež mutiranih potomcev          & 10\%     & 10\%     & 10\%     \\
            \hline
            prispevek kompleksnosti           & -0.00001 & -0.00001 & -0.00001 \\
            \hline
            število generacij                 & 200      & 200      & 300      \\
            \hline
        \end{tabular}
    \end{center}
    \caption{Nabori inicializacijskih parametrov poganjanja na množici Iris.}
    \label{tab:param_iris}
\end{table}

\subsection{Prvi nabor}\label{subsec:dodatek-iris-prvi-nabor}
%%"/home/jure/CLionProjects/Neuroevolution/datasets/iris/iris.data" 100 5 10 2 true 0.1 50 true -0.00001 200 ACC
\begin{table}[H]
    \begin{center}
        \begin{tabular}{|| c | c c || c c ||}
            \hline
            \multirow{2}{*}{št. zagona} & \multicolumn{2}{c||}{točnost najboljšega agenta} & \multicolumn{2}{c||}{MCC najboljšega agenta} \\ \cline{2-5}
            & učna   & testna          & učna  & testna                  \\
            \hline
            1        & 96.2\% & \textbf{93.3\%} & 0.612 & 0.612                   \\
            \hline
            2        & 94.3\% & 93.3\%          & 0.612 & 0.612                   \\
            \hline
            3        & 90.5\% & 88.9\%          & 0.667 & 0.609                   \\
            \hline
            4        & 66.7\% & 66.7\%          & 0.849 & \textbf{0.871 (91.1\%)} \\
            \hline
            5        & 95.2\% & 91.1\%          & 0.612 & 0.552                   \\
            \hline
            $\sigma$ & 0.111  & 0.101           & 0.092 & 0.112                   \\
            \hline
        \end{tabular}
    \end{center}
    \caption{Rezultat prvega nabora parametrov.}
    \label{tab:iris_result_1}
\end{table}

\begin{table}[H]
    \centering
    \begin{tabular}{||rcccc||}
        \hline
        razred           & Iris Setosa & Iris Versicolour & Iris Virginica & vsota \\ \hline
        Iris Setosa      & 15          & 0                & 0              & 15    \\ \hline
        Iris Versicolour & 0           & 13               & 2              & 15    \\ \hline
        Iris Virginica   & 0           & 1                & 14             & 15    \\ \hline
        vsota            & 15          & 14               & 16             & 45    \\ \hline
    \end{tabular}
    \caption{Matrika zmot najbolj točnega agenta prvega nabora.}
    \label{tab:iris_acc_1}
\end{table}

\begin{table}[H]
    \centering
    \begin{tabular}{||rcccc||}
        \hline
        razred           & Iris Setosa & Iris Versicolour & Iris Virginica & vsota \\ \hline
        Iris Setosa      & 15          & 0                & 0              & 15    \\ \hline
        Iris Versicolour & 1           & 14               & 0              & 15    \\ \hline
        Iris Virginica   & 0           & 3                & 12             & 15    \\ \hline
        vsota            & 16          & 17               & 12             & 45    \\ \hline
    \end{tabular}
    \caption{Matrika zmot agenta z največjim MCC prvega nabora.}
    \label{tab:iris_mcc_1}
\end{table}

\begin{figure}[H]
    \begin{center}
        \includegraphics[width=13cm]{iris/1/acc}
    \end{center}
    \caption{Graf točnosti populacije najboljšega agenta prvega nabora skozi generacije.}
    \label{fig:iris_acc_1}
\end{figure}

\begin{figure}[H]
    \begin{center}
        \includegraphics[width=13cm]{iris/1/mcc}
    \end{center}
    \caption{Graf MCC populacije najboljšega agenta prvega nabora skozi generacije.}
    \label{fig:iris_mcc_1}
\end{figure}

\begin{figure}[H]
    \begin{center}
        \includegraphics[width=13cm]{iris/1/acc_g}
    \end{center}
    \caption{Vizualizacija najbolj točnega agenta prvega nabora. Vsebuje 5 povezav.}
    \label{fig:iris_acc_1_g}
\end{figure}

\begin{figure}[H]
    \begin{center}
        \includegraphics[width=13cm]{iris/1/mcc_g}
    \end{center}
    \caption{Vizualizacija agenta z največjim MCC prvega nabora. Vsebuje 5 povezav.}
    \label{fig:iris_mcc_1_g}
\end{figure}

\subsection{Drugi nabor}\label{subsec:dodatek-iris-drugi-nabor}
%%"/home/jure/CLionProjects/Neuroevolution/datasets/iris/iris.data" 150 10 20 2 true 0.1 75 true -0.00001 200 ACC
\begin{table}[H]
    \begin{center}
        \begin{tabular}{|| c | c c || c c ||}
            \hline
            \multirow{2}{*}{št. zagona} & \multicolumn{2}{c||}{točnost najboljšega agenta} & \multicolumn{2}{c||}{MCC najboljšega agenta} \\ \cline{2-5}
            & učna   & testna          & učna  & testna                 \\
            \hline
            1        & 94.3\% & 95.6\%          & 0.957 & 0.906                  \\
            \hline
            2        & 97.1\% & \textbf{97.8\%} & 0.957 & 0.901                  \\
            \hline
            3        & 97.1\% & 91.1\%          & 0.929 & 0.877                  \\
            \hline
            4        & 93.3\% & 97.8\%          & 0.972 & 0.906                  \\
            \hline
            5        & 84.8\% & 88.9\%          & 0.958 & \textbf{1.000 (100\%)} \\
            \hline
            $\sigma$ & 0.045  & 0.036           & 0.014 & 0.042                  \\
            \hline
        \end{tabular}
    \end{center}
    \caption{Rezultat drugega nabora parametrov.}
    \label{tab:iris_result_2}
\end{table}

\begin{table}[H]
    \centering
    \begin{tabular}{||rcccc||}
        \hline
        razred           & Iris Setosa & Iris Versicolour & Iris Virginica & vsota \\ \hline
        ris Setosa       & 15          & 0                & 0              & 15    \\ \hline
        Iris Versicolour & 0           & 15               & 0              & 15    \\ \hline
        Iris Virginica   & 0           & 1                & 14             & 15    \\ \hline
        vsota            & 15          & 16               & 14             & 45    \\ \hline
    \end{tabular}
    \caption{Matrika zmot najbolj točnega agenta drugega nabora.}
    \label{tab:iris_acc_2}
\end{table}

\begin{table}[H]
    \centering
    \begin{tabular}{||rcccc||}
        \hline
        razred           & Iris Setosa & Iris Versicolour & Iris Virginica & vsota \\ \hline
        ris Setosa       & 15          & 0                & 0              & 15    \\ \hline
        Iris Versicolour & 0           & 15               & 0              & 15    \\ \hline
        Iris Virginica   & 0           & 0                & 15             & 15    \\ \hline
        vsota            & 15          & 15               & 15             & 45    \\ \hline
    \end{tabular}
    \caption{Matrika zmot agenta z največjim MCC drugega nabora.}
    \label{tab:iris_mcc_2}
\end{table}

\begin{figure}[H]
    \begin{center}
        \includegraphics[width=13cm]{iris/2/acc}
    \end{center}
    \caption{Graf točnosti populacije najboljšega agenta drugega nabora skozi generacije.}
    \label{fig:iris_acc_2}
\end{figure}

\begin{figure}[H]
    \begin{center}
        \includegraphics[width=13cm]{iris/2/mcc}
    \end{center}
    \caption{Graf MCC populacije najboljšega agenta drugega nabora skozi generacije.}
    \label{fig:iris_mcc_2}
\end{figure}

\begin{figure}[H]
    \begin{center}
        \includegraphics[width=13cm]{iris/2/acc_g}
    \end{center}
    \caption{Vizualizacija najbolj točnega agenta drugega nabora. Vsebuje 1 globoko vozlišče in 9 povezav.}
    \label{fig:iris_acc_2_g}
\end{figure}

\begin{figure}[H]
    \begin{center}
        \includegraphics[width=13cm]{iris/2/mcc_g}
    \end{center}
    \caption{Vizualizacija agenta z največjim MCC drugega nabora. Vsebuje 6 povezav.}
    \label{fig:iris_mcc_2_g}
\end{figure}

\subsection{Tretji nabor}\label{subsec:dodatek-iris-tretji-nabor}
%%"/home/jure/CLionProjects/Neuroevolution/datasets/iris/iris.data" 200 15 30 3 true 0.1 100 true -0.00001 300 ACC
\begin{table}[H]
    \begin{center}
        \begin{tabular}{|| c | c c || c c ||}
            \hline
            \multirow{2}{*}{št. zagona} & \multicolumn{2}{c||}{točnost najboljšega agenta} & \multicolumn{2}{c||}{MCC najboljšega agenta} \\ \cline{2-5}
            & učna   & testna          & učna  & testna                  \\
            \hline
            1        & 94.3\% & \textbf{97.8\%} & 0.972 & 0.906                   \\
            \hline
            2        & 95.2\% & 84.4\%          & 0.986 & 0.936                   \\
            \hline
            3        & 98.1\% & 95.6\%          & 0.972 & 0.839                   \\
            \hline
            4        & 97.1\% & 93.3\%          & 0.958 & 0.936                   \\
            \hline
            5        & 95.2\% & 93.3\%          & 0.972 & \textbf{0.967 (97.8\%)} \\
            \hline
            $\sigma$ & 0.014  & 0.046           & 0.009 & 0.043                   \\
            \hline
        \end{tabular}
    \end{center}
    \caption{Rezultat tretjega nabora parametrov.}
    \label{tab:iris_result_3}
\end{table}

\begin{table}[H]
    \centering
    \begin{tabular}{||rcccc||}
        \hline
        razred           & Iris Setosa & Iris Versicolour & Iris Virginica & vsota \\ \hline
        ris Setosa       & 15          & 0                & 0              & 15    \\ \hline
        Iris Versicolour & 0           & 14               & 1              & 15    \\ \hline
        Iris Virginica   & 0           & 0                & 15             & 15    \\ \hline
        vsota            & 15          & 14               & 16             & 45    \\ \hline
    \end{tabular}
    \caption{Matrika zmot najbolj točnega agenta tretjega nabora.}
    \label{tab:iris_acc_3}
\end{table}

\begin{table}[H]
    \centering
    \begin{tabular}{||rcccc||}
        \hline
        razred           & Iris Setosa & Iris Versicolour & Iris Virginica & vsota \\ \hline
        ris Setosa       & 15          & 0                & 0              & 15    \\ \hline
        Iris Versicolour & 0           & 14               & 1              & 15    \\ \hline
        Iris Virginica   & 0           & 0                & 15             & 15    \\ \hline
        vsota            & 15          & 14               & 16             & 45    \\ \hline
    \end{tabular}
    \caption{Matrika zmot agenta z največjim MCC tretjega nabora.}
    \label{tab:iris_mcc_3}
\end{table}

\begin{figure}[H]
    \begin{center}
        \includegraphics[width=13cm]{iris/3/acc}
    \end{center}
    \caption{Graf točnosti populacije najboljšega agenta tretjega nabora skozi generacije.}
    \label{fig:iris_acc_3}
\end{figure}

\begin{figure}[H]
    \begin{center}
        \includegraphics[width=13cm]{iris/3/mcc}
    \end{center}
    \caption{Graf MCC populacije najboljšega agenta tretjega nabora skozi generacije.}
    \label{fig:iris_mcc_3}
\end{figure}

\begin{figure}[H]
    \begin{center}
        \includegraphics[width=13cm]{iris/3/acc_g}
    \end{center}
    \caption{Vizualizacija najbolj točnega agenta tretjega nabora. Vsebuje 6 povezav.}
    \label{fig:iris_acc_3_g}
\end{figure}

\begin{figure}[H]
    \begin{center}
        \includegraphics[width=13cm]{iris/3/mcc_g}
    \end{center}
    \caption{Vizualizacija agenta z največjim MCC drugega nabora. Vsebuje 6 povezav.}
    \label{fig:iris_mcc_3_g}
\end{figure}

\section{Wine}\label{sec:dodatek-wine-test}
%% arrowLength=20
%% linkWidth=2
%% input fy=50*node.pos
%% output fx=700
%% output fy=150*node.pos+120
%% MAX_FONT_SIZE=12
\begin{table}[H]
    \begin{center}
        \begin{tabular}{||l c c c||}
            \hline
            & 1      & 2      & 3 \\ [0.5ex]
            \hline
            velikost populacije               & 200    & 250    & 350    \\
            \hline
            največje število globokih vozlišč & 15     & 20     & 25     \\
            \hline
            največje število povezav          & 30     & 50     & 75     \\
            \hline
            največje število prečkanj         & 2      & 3      & 4      \\
            \hline
            delež mutiranih potomcev          & 10\%   & 10\%   & 10\%   \\
            \hline
            prispevek vozlišč                 & -0.001 & -0.001 & -0.001 \\
            \hline
            prispevek povezav                 & -0.001 & -0.001 & -0.001 \\
            \hline
            število generacij                 & 300    & 350    & 450    \\
            \hline
        \end{tabular}
    \end{center}
    \caption{Nabori inicializacijskih parametrov poganjanja na množici Wine.}
    \label{tab:param_wine}
\end{table}

\subsection{Prvi nabor}\label{subsec:dodatek-wine-prvi-nabor}
%%"/home/jure/CLionProjects/Neuroevolution/datasets/wine/wine.data" 200 15 30 2 true 0.1 100 true -0.001 -0.001 300 ACC
\begin{table}[H]
    \begin{center}
        \begin{tabular}{|| c | c c || c c ||}
            \hline
            \multirow{2}{*}{št. zagona} & \multicolumn{2}{c||}{točnost najboljšega agenta} & \multicolumn{2}{c||}{MCC najboljšega agenta} \\ \cline{2-5}
            & učna   & testna          & učna  & testna                  \\
            \hline
            1        & 92.0\% & 75.5\%          & 0.868 & \textbf{0.918 (94.3\%)} \\
            \hline
            2        & 94.4\% & 92.5\%          & 0.808 & 0.843                   \\
            \hline
            3        & 92.0\% & 92.5\%          & 0.818 & 0.779                   \\
            \hline
            4        & 96.8\% & \textbf{94.3\%} & 0.820 & 0.677                   \\
            \hline
            5        & 84.8\% & 75.5\%          & 0.904 & 0.830                   \\
            \hline
            $\sigma$ & 0.040  & 0.086           & 0.037 & 0.08                    \\
            \hline
        \end{tabular}
    \end{center}
    \caption{Rezultat prvega nabora parametrov.}
    \label{tab:wine_result_1}
\end{table}

\begin{table}[H]
    \centering
    \begin{tabular}{||rcccc||}
        \hline
        razred  & Class 1 & Class 2 & Class 3 & vsota \\ \hline
        Class 1 & 17      & 1       & 0       & 18    \\ \hline
        Class 2 & 2       & 19      & 0       & 21    \\ \hline
        Class 3 & 0       & 0       & 14      & 14    \\ \hline
        vsota   & 19      & 20      & 14      & 53    \\ \hline
    \end{tabular}
    \caption{Matrika zmot najbolj točnega agenta prvega nabora.}
    \label{tab:wine_acc_1}
\end{table}

\begin{table}[H]
    \centering
    \begin{tabular}{||rcccc||}
        \hline
        razred  & Class 1 & Class 2 & Class 3 & vsota \\ \hline
        Class 1 & 18      & 0       & 0       & 18    \\ \hline
        Class 2 & 2       & 18      & 1       & 21    \\ \hline
        Class 3 & 0       & 0       & 14      & 14    \\ \hline
        vsota   & 20      & 18      & 15      & 53    \\ \hline
    \end{tabular}
    \caption{Matrika zmot agenta z največjim MCC prvega nabora.}
    \label{tab:wine_mcc_1}
\end{table}

\begin{figure}[H]
    \begin{center}
        \includegraphics[width=13cm]{wine/1/acc}
    \end{center}
    \caption{Graf točnosti populacije najboljšega agenta prvega nabora skozi generacije.}
    \label{fig:wine_acc_1}
\end{figure}

\begin{figure}[H]
    \begin{center}
        \includegraphics[width=13cm]{wine/1/mcc}
    \end{center}
    \caption{Graf MCC populacije najboljšega agenta prvega nabora skozi generacije.}
    \label{fig:wine_mcc_1}
\end{figure}

\begin{figure}[H]
    \begin{center}
        \includegraphics[width=13cm]{wine/1/acc_g}
    \end{center}
    \caption{Vizualizacija najbolj točnega agenta prvega nabora. Vsebuje 1 globoko vozlišče in 9 povezav.}
    \label{fig:wine_acc_1_g}
\end{figure}

\begin{figure}[H]
    \begin{center}
        \includegraphics[width=13cm]{wine/1/mcc_g}
    \end{center}
    \caption{Vizualizacija agenta z največjim MCC prvega nabora. Vsebuje 6 povezav.}
    \label{fig:wine_mcc_1_g}
\end{figure}

\subsection{Drugi nabor}\label{subsec:dodatek-wine-drugi-nabor}
%%"/home/jure/CLionProjects/Neuroevolution/datasets/iris/iris.data" 250 20 50 3 true 0.1 100 true -0.001 -0.001 350 ACC
\begin{table}[H]
    \begin{center}
        \begin{tabular}{|| c | c c || c c ||}
            \hline
            \multirow{2}{*}{št. zagona} & \multicolumn{2}{c||}{točnost najboljšega agenta} & \multicolumn{2}{c||}{MCC najboljšega agenta} \\ \cline{2-5}
            & učna   & testna          & učna  & testna                  \\
            \hline
            1        & 83.2\% & 81.1\%          & 0.845 & 0.610                   \\
            \hline
            2        & 67.2\% & 60.4\%          & 0.867 & 0.779                   \\
            \hline
            3        & 76.8\% & 77.4\%          & 0.830 & 0.740                   \\
            \hline
            4        & 92.0\% & \textbf{96.2\%} & 0.915 & \textbf{0.915 (94.3\%)} \\
            \hline
            5        & 93.6\% & 96.2\%          & 0.892 & 0.914                   \\
            \hline
            $\sigma$ & 0.098  & 0.134           & 0.031 & 0.115                   \\
            \hline
        \end{tabular}
    \end{center}
    \caption{Rezultat drugega nabora parametrov.}
    \label{tab:wine_result_2}
\end{table}

\begin{table}[H]
    \centering
    \begin{tabular}{||rcccc||}
        \hline
        razred  & Class 1 & Class 2 & Class 3 & vsota \\ \hline
        Class 1 & 17      & 1       & 0       & 18    \\ \hline
        Class 2 & 0       & 20      & 1       & 21    \\ \hline
        Class 3 & 0       & 0       & 14      & 14    \\ \hline
        vsota   & 17      & 21      & 15      & 53    \\ \hline
    \end{tabular}
    \caption{Matrika zmot najbolj točnega agenta drugega nabora.}
    \label{tab:wine_acc_2}
\end{table}

\begin{table}[H]
    \centering
    \begin{tabular}{||rcccc||}
        \hline
        razred  & Class 1 & Class 2 & Class 3 & vsota \\ \hline
        Class 1 & 17      & 1       & 0       & 18    \\ \hline
        Class 2 & 2       & 19      & 0       & 21    \\ \hline
        Class 3 & 0       & 0       & 14      & 14    \\ \hline
        vsota   & 19      & 20      & 14      & 53    \\ \hline
    \end{tabular}
    \caption{Matrika zmot agenta z največjim MCC drugega nabora.}
    \label{tab:wine_mcc_2}
\end{table}

\begin{figure}[H]
    \begin{center}
        \includegraphics[width=13cm]{wine/2/acc}
    \end{center}
    \caption{Graf točnosti populacije najboljšega agenta drugega nabora skozi generacije.}
    \label{fig:wine_acc_2}
\end{figure}

\begin{figure}[H]
    \begin{center}
        \includegraphics[width=13cm]{wine/2/mcc}
    \end{center}
    \caption{Graf MCC populacije najboljšega agenta drugega nabora skozi generacije.}
    \label{fig:wine_mcc_2}
\end{figure}

\begin{figure}[H]
    \begin{center}
        \includegraphics[width=13cm]{wine/2/acc_g}
    \end{center}
    \caption{Vizualizacija najbolj točnega agenta drugega nabora. Vsebuje 6 povezav.}
    \label{fig:wine_acc_2_g}
\end{figure}

\begin{figure}[H]
    \begin{center}
        \includegraphics[width=13cm]{wine/2/mcc_g}
    \end{center}
    \caption{Vizualizacija agenta z največjim MCC drugega nabora. Vsebuje 1 globoko vozlišče in 7 povezav.}
    \label{fig:wine_mcc_2_g}
\end{figure}

\subsection{Tretji nabor}\label{subsec:dodatek-wine-tretji-nabor}
%%"/home/jure/CLionProjects/Neuroevolution/datasets/iris/iris.data" 350 25 75 4 true 0.1 175 true -0.001 -0.001 450 ACC
\begin{table}[H]
    \begin{center}
        \begin{tabular}{|| c | c c || c c ||}
            \hline
            \multirow{2}{*}{št. zagona} & \multicolumn{2}{c||}{točnost najboljšega agenta} & \multicolumn{2}{c||}{MCC najboljšega agenta} \\ \cline{2-5}
            & učna   & testna          & učna  & testna                  \\
            \hline
            1        & 80.0\% & 60.4\%          & 0.930 & 0.857                   \\
            \hline
            2        & 93.6\% & \textbf{96.2\%} & 0.906 & 0.860                   \\
            \hline
            3        & 95.2\% & 90.6\%          & 0.940 & 0.829                   \\
            \hline
            4        & 93.6\% & 88.7\%          & 0.774 & 0.763                   \\
            \hline
            5        & 88.0\% & 86.8\%          & 0.940 & \textbf{0.887 (92.5\%)} \\
            \hline
            $\sigma$ & 0.056  & 0.125           & 0.063 & 0.042                   \\
            \hline
        \end{tabular}
    \end{center}
    \caption{Rezultat tretjega nabora parametrov.}
    \label{tab:wine_result_3}
\end{table}

\begin{table}[H]
    \centering
    \begin{tabular}{||rcccc||}
        \hline
        razred  & Class 1 & Class 2 & Class 3 & vsota \\ \hline
        Class 1 & 18      & 0       & 0       & 18    \\ \hline
        Class 2 & 2       & 19      & 0       & 21    \\ \hline
        Class 3 & 0       & 0       & 14      & 14    \\ \hline
        vsota   & 20      & 19      & 14      & 53    \\ \hline
    \end{tabular}
    \caption{Matrika zmot najbolj točnega agenta tretjega nabora.}
    \label{tab:wine_acc_3}
\end{table}

\begin{table}[H]
    \centering
    \begin{tabular}{||rcccc||}
        \hline
        razred  & Class 1 & Class 2 & Class 3 & vsota \\ \hline
        Class 1 & 15      & 3       & 0       & 18    \\ \hline
        Class 2 & 1       & 20      & 0       & 21    \\ \hline
        Class 3 & 0       & 0       & 14      & 14    \\ \hline
        vsota   & 16      & 23      & 14      & 53    \\ \hline
    \end{tabular}
    \caption{Matrika zmot agenta z največjim MCC tretjega nabora.}
    \label{tab:wine_mcc_3}
\end{table}

\begin{figure}[H]
    \begin{center}
        \includegraphics[width=13cm]{wine/3/acc}
    \end{center}
    \caption{Graf točnosti populacije najboljšega agenta tretjega nabora skozi generacije.}
    \label{fig:wine_acc_3}
\end{figure}

\begin{figure}[H]
    \begin{center}
        \includegraphics[width=13cm]{wine/3/mcc}
    \end{center}
    \caption{Graf MCC populacije najboljšega agenta tretjega nabora skozi generacije.}
    \label{fig:wine_mcc_3}
\end{figure}

\begin{figure}[H]
    \begin{center}
        \includegraphics[width=13cm]{wine/3/acc_g}
    \end{center}
    \caption{Vizualizacija najbolj točnega agenta tretjega nabora. Vsebuje 11 povezav.}
    \label{fig:wine_acc_3_g}
\end{figure}

\begin{figure}[H]
    \begin{center}
        \includegraphics[width=13cm]{wine/3/mcc_g}
    \end{center}
    \caption{Vizualizacija agenta z največjim MCC tretjega nabora. Vsebuje 2 globoki vozlišči in 7 povezav.}
    \label{fig:wine_mcc_3_g}
\end{figure}

\section{Car Evaluation}\label{sec:dodatek-car-test}
%% arrowLength=10
%% linkWidth=3
%% input fy=50*node.pos
%% output fx=350
%% output fy=50*node.pos+50
%% MAX_FONT_SIZE=8
\begin{table}[H]
    \begin{center}
        \begin{tabular}{||l c c c||}
            \hline
            & 1        & 2        & 3 \\ [0.5ex]
            \hline
            velikost populacije               & 200      & 250      & 350      \\
            \hline
            največje število globokih vozlišč & 15       & 20       & 40       \\
            \hline
            največje število povezav          & 30       & 50       & 100      \\
            \hline
            največje število prečkanj         & 2        & 3        & 4        \\
            \hline
            delež mutiranih potomcev          & 10\%     & 10\%     & 10\%     \\
            \hline
            prispevek vozlišč                 & -0.00001 & -0.00001 & -0.00001 \\
            \hline
            prispevek povezav                 & -0.00001 & -0.00001 & -0.00001 \\
            \hline
            število generacij                 & 200      & 200      & 300      \\
            \hline
        \end{tabular}
    \end{center}
    \caption{Nabori inicializacijskih parametrov poganjanja na množici Car Evaluation.}
    \label{tab:param_car}
\end{table}

\subsubsection{Prvi nabor}
%%"/home/jure/CLionProjects/Neuroevolution/datasets/car/car.data" 200 15 30 2 true 0.1 100 true -0.00001 -0.00001 300 ACC
\begin{table}[H]
    \begin{center}
        \begin{tabular}{|| c | c c || c c ||}
            \hline
            \multirow{2}{*}{št. zagona} & \multicolumn{2}{c||}{točnost najboljšega agenta} & \multicolumn{2}{c||}{MCC najboljšega agenta} \\ \cline{2-5}
            & učna   & testna          & učna  & testna                  \\
            \hline
            1        & 72.0\% & 72.2\%          & 0.499 & 0.474                   \\
            \hline
            2        & 73.1\% & 71.8\%          & 0.504 & 0.461                   \\
            \hline
            3        & 75.0\% & 72.8\%          & 0.489 & 0.498                   \\
            \hline
            4        & 72.6\% & \textbf{73.2\%} & 0.482 & \textbf{0.509 (69.9\%)} \\
            \hline
            5        & 72.6\% & 73.2\%          & 0.479 & 0.428                   \\
            \hline
            $\sigma$ & 0.010  & 0.006           & 0.010 & 0.029                   \\
            \hline
        \end{tabular}
    \end{center}
    \caption{Rezultat prvega nabora parametrov.}
    \label{tab:car_result_1}
\end{table}

\begin{table}[H]
    \centering
    \begin{tabular}{||rccccc||}
        \hline
        razred       & unacceptable & acceptable & good & very good & vsota \\ \hline
        unacceptable & 357          & 6          & 0    & 0         & 363   \\ \hline
        acceptable   & 97           & 18         & 0    & 0         & 115   \\ \hline
        good         & 13           & 8          & 0    & 0         & 21    \\ \hline
        very good    & 9            & 10         & 0    & 0         & 19    \\ \hline
        vsota        & 476          & 42         & 0    & 0         & 518   \\ \hline
    \end{tabular}
    \caption{Matrika zmot najbolj točnega agenta prvega nabora. Agent lahko napove samo razreda \enquote{nesprejemljivo} in \enquote{sprejemljivo}.}
    \label{tab:car_acc_1}
\end{table}

\begin{table}[H]
    \centering
    \caption{Matrika zmot agenta z največjim MCC prvega nabora. Agent lahko napove samo razreda \enquote{nesprejemljivo} in \enquote{sprejemljivo}.}
    \begin{tabular}{||rccccc||}
        \hline
        razred       & unacceptable & acceptable & good & very good & vsota \\ \hline
        unacceptable & 260          & 103        & 0    & 0         & 363   \\ \hline
        acceptable   & 10           & 105        & 0    & 0         & 115   \\ \hline
        good         & 0            & 21         & 0    & 0         & 21    \\ \hline
        very good    & 0            & 19         & 0    & 0         & 19    \\ \hline
        vsota        & 270          & 248        & 0    & 0         & 518   \\ \hline
    \end{tabular}
    \label{tab:car_mcc_1}
\end{table}

\begin{figure}[H]
    \begin{center}
        \includegraphics[width=13cm]{car/1/acc}
    \end{center}
    \caption{Graf točnosti populacije najboljšega agenta prvega nabora skozi generacije.}
    \label{fig:car_acc_1}
\end{figure}

\begin{figure}[H]
    \begin{center}
        \includegraphics[width=13cm]{car/1/mcc}
    \end{center}
    \caption{Graf MCC populacije najboljšega agenta prvega nabora skozi generacije.}
    \label{fig:car_mcc_1}
\end{figure}

\begin{figure}[H]
    \begin{center}
        \includegraphics[width=13cm]{car/1/acc_g}
    \end{center}
    \caption{Vizualizacija najbolj točnega agenta prvega nabora. Vsebuje 6 globokih vozlišč in 25 povezav.}
    \label{fig:car_acc_1_g}
\end{figure}

\begin{figure}[H]
    \begin{center}
        \includegraphics[width=13cm]{car/1/mcc_g}
    \end{center}
    \caption{Vizualizacija agenta z največjim MCC prvega nabora. Vsebuje 5 globokih vozlišč in 23 povezav.}
    \label{fig:car_mcc_1_g}
\end{figure}

\subsubsection{Drugi nabor}
%% 250 20 50 3 true 0.1 125 true -0.00001 -0.00001 200 ACC
\begin{table}[H]
    \caption{Rezultat drugega nabora parametrov.}
    \begin{center}
        \begin{tabular}{|| c | c c || c c ||}
            \hline
            \multirow{2}{*}{št. zagona} & \multicolumn{2}{c||}{točnost najboljšega agenta} & \multicolumn{2}{c||}{MCC najboljšega agenta} \\ \cline{2-5}
            & učna    & testna           & učna  & testna         \\
            \hline
            1        & 0.729\% & 0.716\%          & 0.505 & 0.471          \\
            \hline
            2        & 0.716\% & 0.718\%          & 0.486 & 0.507          \\
            \hline
            3        & 0.707\% & 0.701\%          & 0.475 & \textbf{0.530} \\
            \hline
            4        & 0.719\% & \textbf{0.724\%} & 0.482 & 0.513          \\
            \hline
            5        & 0.713\% & 0.695\%          & 0.501 & 0.479          \\
            \hline
            $\sigma$ & 0.007   & 0.011            & 0.011 & 0.022          \\
            \hline
        \end{tabular}
    \end{center}
    \label{tab:car_result_2}
\end{table}

\begin{table}[H]
    \centering
    \caption{Matrika zmot najbolj točnega agenta drugega nabora. Agent lahko napove samo razreda \enquote{nesprejemljivo} in \enquote{zelo dobro}.}
    \begin{tabular}{||rccccc||}
        \hline
        razred       & unacceptable & acceptable & good & very good & vsota \\ \hline
        unacceptable & 361          & 0          & 0    & 2         & 363   \\ \hline
        acceptable   & 102          & 0          & 0    & 13        & 115   \\ \hline
        good         & 15           & 0          & 0    & 6         & 21    \\ \hline
        very good    & 5            & 0          & 0    & 14        & 19    \\ \hline
        vsota        & 473          & 0          & 0    & 35        & 518   \\ \hline
    \end{tabular}
    \label{tab:car_acc_2}
\end{table}

\begin{table}[H]
    \centering
    \caption{Matrika zmot agenta z največjim MCC drugega nabora. Agent lahko napove samo razreda \enquote{nesprejemljivo} in \enquote{sprejemljivo}.}
    \begin{tabular}{||rccccc||}
        \hline
        razred       & unacceptable & acceptable & good & very good & vsota \\ \hline
        unacceptable & 276          & 87         & 0    & 0         & 363   \\ \hline
        acceptable   & 9            & 106        & 0    & 0         & 115   \\ \hline
        good         & 0            & 21         & 0    & 0         & 21    \\ \hline
        very good    & 0            & 19         & 0    & 0         & 19    \\ \hline
        vsota        & 285          & 233        & 0    & 0         & 518   \\ \hline
    \end{tabular}
    \label{tab:car_mcc_2}
\end{table}

\begin{figure}[H]
    \begin{center}
        \includegraphics[width=13cm]{car/2/acc}
    \end{center}
    \caption{Graf točnosti populacije najboljšega agenta drugega nabora skozi generacije.}
    \label{fig:car_acc_2}
\end{figure}

\begin{figure}[H]
    \begin{center}
        \includegraphics[width=13cm]{car/2/mcc}
    \end{center}
    \caption{Graf MCC populacije najboljšega agenta drugega nabora skozi generacije.}
    \label{fig:car_mcc_2}
\end{figure}

\begin{figure}[H]
    \begin{center}
        \includegraphics[width=13cm]{car/2/acc_g}
    \end{center}
    \caption{Vizualizacija najbolj točnega agenta drugega nabora. Vsebuje 5 globokih vozlišč in 36 povezav.}
    \label{fig:car_acc_2_g}
\end{figure}

\begin{figure}[H]
    \begin{center}
        \includegraphics[width=13cm]{car/2/mcc_g}
    \end{center}
    \caption{Vizualizacija agenta z največjim MCC drugega nabora. Vsebuje 8 globokih vozlišč in 39 povezav.}
    \label{fig:car_mcc_2_g}
\end{figure}

\subsubsection{Tretji nabor}
%% 350 40 100 4 true 0.1 175 true -0.00001 -0.00001 300 ACC
\begin{table}[H]
    \caption{Rezultat tretjega nabora parametrov.}
    \begin{center}
        \begin{tabular}{|| c | c c || c c ||}
            \hline
            \multirow{2}{*}{št. zagona} & \multicolumn{2}{c||}{točnost najboljšega agenta} & \multicolumn{2}{c||}{MCC najboljšega agenta} \\ \cline{2-5}
            & učna    & testna           & učna  & testna         \\
            \hline
            1        & 0.716\% & 0.708\%          & 0.471 & 0.480          \\
            \hline
            2        & 0.719\% & 0.705\%          & 0.502 & 0.473          \\
            \hline
            3        & 0.729\% & 0.728\%          & 0.482 & \textbf{0.486} \\
            \hline
            4        & 0.740\% & 0.716\%          & 0.512 & 0.443          \\
            \hline
            5        & 0.737\% & \textbf{0.734\%} & 0.501 & 0.479          \\
            \hline
            $\sigma$ & 0.009   & 0.011            & 0.015 & 0.015          \\
            \hline
        \end{tabular}
    \end{center}
    \label{tab:car_result_3}
\end{table}

\begin{table}[H]
    \centering
    \caption{Matrika zmot najbolj točnega agenta tretjega nabora. Agent lahko napove samo razreda \enquote{nesprejemljivo} in \enquote{sprejemljivo}.}
    \begin{tabular}{||rccccc||}
        \hline
        razred       & unacceptable & acceptable & good & very good & vsota \\ \hline
        unacceptable & 306          & 57         & 0    & 0         & 363   \\ \hline
        acceptable   & 41           & 74         & 0    & 0         & 115   \\ \hline
        good         & 0            & 21         & 0    & 0         & 21    \\ \hline
        very good    & 0            & 19         & 0    & 0         & 19    \\ \hline
        vsota        & 347          & 171        & 0    & 0         & 518   \\ \hline
    \end{tabular}
    \label{tab:car_acc_3}
\end{table}

\begin{table}[H]
    \centering
    \caption{Matrika zmot agenta z največjim MCC tretjega nabora. Agent lahko pravilno napove samo razreda \enquote{nesprejemljivo} in \enquote{sprejemljivo}.}
    \begin{tabular}{||rccccc||}
        \hline
        razred       & unacceptable & acceptable & good & very good & vsota \\ \hline
        unacceptable & 267          & 88         & 0    & 8         & 363   \\ \hline
        acceptable   & 13            & 102        & 0    & 0         & 115   \\ \hline
        good         & 0            & 21         & 0    & 0         & 21    \\ \hline
        very good    & 0            & 19         & 0    & 0         & 19    \\ \hline
        vsota        & 280          & 230        & 0    & 8         & 518   \\ \hline
    \end{tabular}
    \label{tab:car_mcc_3}
\end{table}

\begin{figure}[H]
    \begin{center}
        \includegraphics[width=13cm]{car/3/acc}
    \end{center}
    \caption{Graf točnosti populacije najboljšega agenta tretjega nabora skozi generacije.}
    \label{fig:car_acc_3}
\end{figure}

\begin{figure}[H]
    \begin{center}
        \includegraphics[width=13cm]{car/3/mcc}
    \end{center}
    \caption{Graf MCC populacije najboljšega agenta tretjega nabora skozi generacije.}
    \label{fig:car_mcc_3}
\end{figure}

\begin{figure}[H]
    \begin{center}
        \includegraphics[width=13cm]{car/3/acc_g}
    \end{center}
    \caption{Vizualizacija najbolj točnega agenta tretjega nabora. Vsebuje 15 globokih vozlišč in 64 povezav.}
    \label{fig:car_acc_3_g}
\end{figure}

\begin{figure}[H]
    \begin{center}
        \includegraphics[width=13cm]{car/3/mcc_g}
    \end{center}
    \caption{Vizualizacija agenta z največjim MCC drugega nabora. Vsebuje 15 globokih vozlišč in 67 povezav.}
    \label{fig:car_mcc_3_g}
\end{figure}

\section{Shuttle}\label{sec:dodatek-statlog-test}
%% arrowLength=10
%% linkWidth=3
%% input fy=50*node.pos
%% output fx=550
%% output fy=50*node.pos+25
%% MAX_FONT_SIZE=8
Imena razredov v glavah matrik zmot so okrajšana zaradi formatiranja dokumenta.
\begin{table}[H]
    \begin{center}
        \begin{tabular}{||l c c c||}
            \hline
            & 1        & 2        & 3 \\ [0.5ex]
            \hline
            velikost populacije              & 200      & 250      & 350      \\
            \hline
            največje število vmesnih vozlišč & 15       & 20       & 40       \\
            \hline
            največje število povezav         & 30       & 50       & 100      \\
            \hline
            največje število prečkanj        & 2        & 3        & 4        \\
            \hline
            delež mutiranih potomcev         & 10\%     & 10\%     & 10\%     \\
            \hline
            prispevek kompleksnosti          & -0.00001 & -0.00001 & -0.00001 \\
            \hline
            število generacij                & 200      & 250      & 300      \\
            \hline
        \end{tabular}
    \end{center}
    \caption{Nabori inicializacijskih parametrov poganjanja na množici Shuttle.}
    \label{tab:param_statlog}
\end{table}

\subsection{Prvi nabor}\label{subsec:dodatek-statlog-prvi-nabor}
%% branch shuttle
%% 200 15 30 2 true 0.1 100 true -0.00001 200 ACC
\begin{table}[H]
    \begin{center}
        \begin{tabular}{|| c | c c || c c ||}
            \hline
            \multirow{2}{*}{št. zagona} & \multicolumn{2}{c||}{točnost najboljšega agenta} & \multicolumn{2}{c||}{MKK najboljšega agenta} \\ \cline{2-5}
            & učna   & testna          & učna  & testna                  \\
            \hline
            1         & 91.6\% & 92.1\%          & 0.753 & 0.760                   \\
            \hline
            2         & 91.7\% & \textbf{92.2\%} & 0.750 & 0.758                   \\
            \hline
            3         & 91.7\% & 92.1\%          & 0.749 & 0.758                   \\
            \hline
            4         & 89.0\% & 89.5\%          & 0.749 & 0.756                   \\
            \hline
            5         & 89.3\% & 89.8\%          & 0.758 & \textbf{0.766 (92.2\%)} \\
            \hline
            povprečje & 90.7\% & 91.1\%          & 0.752 & 0.760                   \\
            \hline
            $\sigma$  & 0.012  & 0.012           & 0.003 & 0.003                   \\
            \hline
        \end{tabular}
    \end{center}
    \caption{Rezultat prvega nabora parametrov.}
    \label{tab:statlog_result_1}
\end{table}

%\begin{table}[H]
%    \centering
%    \begin{tabular}{||rcccccccc||}
%        \hline
%        razred    & RF    & FC & FO & High & Bypass & BC & BO & vsota \\ \hline
%        Rad Flow  & 11403 & 3  & 0  & 72   & 0      & 0  & 0  & 11478 \\ \hline
%        Fpv Close & 1     & 10 & 0  & 2    & 0      & 0  & 0  & 13    \\ \hline
%        Fpv Open  & 21    & 0  & 0  & 16   & 2      & 0  & 0  & 39    \\ \hline
%        High      & 1006  & 0  & 0  & 1149 & 0      & 0  & 0  & 2155  \\ \hline
%        Bypass    & 1     & 6  & 0  & 1    & 801    & 0  & 0  & 809   \\ \hline
%        Bpv Close & 0     & 4  & 0  & 0    & 0      & 0  & 0  & 4     \\ \hline
%        Bpv Open  & 2     & 0  & 0  & 0    & 0      & 0  & 0  & 2     \\ \hline
%        vsota     & 12434 & 23 & 0  & 1240 & 803    & 0  & 0  & 14500 \\ \hline
%    \end{tabular}
%    \caption{Matrika zmot najbolj točnega agenta prvega nabora. Agent ne more napovedati razredov \enquote{Fpv Open}, \enquote{Bpv Close} in \enquote{Bpv Open}.}
%    \label{tab:statlog_acc_1}
%\end{table}
%
%\begin{table}[H]
%    \centering
%    \begin{tabular}{||rcccccccc||}
%        \hline
%        razred    & RF    & FC & FO & High & Bypass & BC & BO & vsota \\ \hline
%        Rad Flow  & 11442 & 1  & 2  & 33   & 0      & 0  & 0  & 11478 \\ \hline
%        Fpv Close & 8     & 0  & 0  & 4    & 1      & 0  & 0  & 13    \\ \hline
%        Fpv Open  & 24    & 0  & 0  & 13   & 1      & 0  & 1  & 39    \\ \hline
%        High      & 1023  & 0  & 0  & 1132 & 0      & 0  & 0  & 2155  \\ \hline
%        Bypass    & 0     & 1  & 3  & 4    & 800    & 0  & 1  & 809   \\ \hline
%        Bpv Close & 1     & 0  & 0  & 3    & 0      & 0  & 0  & 4     \\ \hline
%        Bpv Open  & 2     & 0  & 0  & 0    & 0      & 0  & 0  & 2     \\ \hline
%        vsota     & 12500 & 2  & 5  & 1189 & 802    & 0  & 2  & 14500 \\ \hline
%    \end{tabular}
%    \caption{Matrika zmot agenta z največjim MKK prvega nabora. Agent ne more napovedati razreda \enquote{Bpv Close}.}
%    \label{tab:statlog_mcc_1}
%\end{table}

\begin{figure}[H]
    \begin{center}
        \includegraphics[width=13cm]{shuttle/1/acc}
    \end{center}
    \caption{Graf točnosti populacije najboljšega agenta prvega nabora skozi generacije.}
    \label{fig:statlog_acc_1}
\end{figure}

\begin{figure}[H]
    \begin{center}
        \includegraphics[width=13cm]{shuttle/1/mcc}
    \end{center}
    \caption{Graf MKK populacije najboljšega agenta prvega nabora skozi generacije.}
    \label{fig:statlog_mcc_1}
\end{figure}

\begin{figure}[H]
    \begin{center}
        \includegraphics[width=13cm]{shuttle/1/acc_g}
    \end{center}
    \caption{Vizualizacija najbolj točnega agenta prvega nabora. Vsebuje 12 povezav.}
    \label{fig:statlog_acc_1_g}
\end{figure}

\begin{figure}[H]
    \begin{center}
        \includegraphics[width=13cm]{shuttle/1/mcc_g}
    \end{center}
    \caption{Vizualizacija agenta z največjim MKK prvega nabora. Vsebuje 16 povezav.}
    \label{fig:statlog_mcc_1_g}
\end{figure}

\subsection{Drugi nabor}\label{subsec:dodatek-statlog-drugi-nabor}
%% 250 20 50 3 true 0.1 125 true -0.00001 250 ACC
\begin{table}[H]
    \begin{center}
        \begin{tabular}{|| c | c c || c c ||}
            \hline
            \multirow{2}{*}{št. zagona} & \multicolumn{2}{c||}{točnost najboljšega agenta} & \multicolumn{2}{c||}{MKK najboljšega agenta} \\ \cline{2-5}
            & učna   & testna          & učna  & testna                  \\
            \hline
            1         & 87.6\% & 88.1\%          & 0.728 & 0.735                   \\
            \hline
            2         & 92.0\% & \textbf{92.4\%} & 0.600 & 0.604                   \\
            \hline
            3         & 86.4\% & 86.9\%          & 0.622 & 0.622                   \\
            \hline
            4         & 86.6\% & 87.1\%          & 0.767 & \textbf{0.774 (92.5\%)} \\
            \hline
            5         & 91.7\% & 92.2\%          & 0.755 & 0.761                   \\
            \hline
            povprečje & 88.9\% & 89.3\%          & 0.694 & 0.699                   \\
            \hline
            $\sigma$  & 0.025  & 0.022           & 0.070 & 0.072                   \\
            \hline
        \end{tabular}
    \end{center}
    \caption{Rezultat drugega nabora parametrov.}
    \label{tab:statlog_result_2}
\end{table}

%\begin{table}[H]
%    \centering
%    \begin{tabular}{||rcccccccc||}
%        \hline
%        razred    & RF    & FC & FO & High & Bypass & BC & BO & vsota \\ \hline
%        Rad Flow  & 11343 & 0  & 0  & 122  & 0      & 3  & 10 & 11478 \\ \hline
%        Fpv Close & 5     & 0  & 0  & 8    & 0      & 0  & 0  & 13    \\ \hline
%        Fpv Open  & 25    & 0  & 0  & 12   & 2      & 0  & 0  & 39    \\ \hline
%        High      & 907   & 0  & 0  & 1248 & 0      & 0  & 0  & 2155  \\ \hline
%        Bypass    & 0     & 0  & 0  & 0    & 805    & 3  & 1  & 809   \\ \hline
%        Bpv Close & 0     & 0  & 0  & 3    & 1      & 0  & 0  & 4     \\ \hline
%        Bpv Open  & 0     & 0  & 0  & 0    & 0      & 0  & 2  & 2     \\ \hline
%        vsota     & 12280 & 0  & 0  & 1393 & 808    & 6  & 13 & 14500 \\ \hline
%    \end{tabular}
%    \caption{Matrika zmot najbolj točnega agenta drugega nabora. Agent ne more napovedati razredov \enquote{Fpv Close} in \enquote{Fpv Open}.}
%    \label{tab:statlog_acc_2}
%\end{table}
%
%\begin{table}[H]
%    \centering
%    \begin{tabular}{||rcccccccc||}
%        \hline
%        razred    & RF    & FC & FO & High & Bypass & BC & BO & vsota \\ \hline
%        Rad Flow  & 11416 & 0  & 1  & 51   & 0      & 4  & 6  & 11478 \\ \hline
%        Fpv Close & 6     & 0  & 0  & 7    & 0      & 0  & 0  & 13    \\ \hline
%        Fpv Open  & 17    & 1  & 0  & 21   & 0      & 0  & 0  & 39    \\ \hline
%        High      & 954   & 0  & 0  & 1200 & 0      & 0  & 1  & 2155  \\ \hline
%        Bypass    & 0     & 1  & 4  & 4    & 797    & 2  & 1  & 809   \\ \hline
%        Bpv Close & 0     & 0  & 4  & 0    & 0      & 0  & 0  & 4     \\ \hline
%        Bpv Open  & 2     & 0  & 0  & 0    & 0      & 0  & 0  & 2     \\ \hline
%        vsota     & 12395 & 2  & 9  & 1283 & 797    & 6  & 8  & 14500 \\ \hline
%    \end{tabular}
%    \caption{Matrika zmot agenta z največjim MKK drugega nabora.}
%    \label{tab:statlog_mcc_2}
%\end{table}

\begin{figure}[H]
    \begin{center}
        \includegraphics[width=13cm]{shuttle/2/acc}
    \end{center}
    \caption{Graf točnosti populacije najboljšega agenta drugega nabora skozi generacije.}
    \label{fig:statlog_acc_2}
\end{figure}

\begin{figure}[H]
    \begin{center}
        \includegraphics[width=13cm]{shuttle/2/mcc}
    \end{center}
    \caption{Graf MKK populacije najboljšega agenta drugega nabora skozi generacije.}
    \label{fig:statlog_mcc_2}
\end{figure}

\begin{figure}[H]
    \begin{center}
        \includegraphics[width=13cm]{shuttle/2/acc_g}
    \end{center}
    \caption{Vizualizacija najbolj točnega agenta drugega nabora. Vsebuje 2 vmesni vozlišči in 26 povezav.}
    \label{fig:statlog_acc_2_g}
\end{figure}

\begin{figure}[H]
    \begin{center}
        \includegraphics[width=13cm]{shuttle/2/mcc_g}
    \end{center}
    \caption{Vizualizacija agenta z največjim MKK drugega nabora. Vsebuje 1 vmesno vozlišče in 18 povezav.}
    \label{fig:statlog_mcc_2_g}
\end{figure}

\subsection{Tretji nabor}\label{subsec:dodatek-statlog-tretji-nabor}
%% 350 40 100 4 true 0.1 175 true -0.00001 300 ACC
\begin{table}[H]
    \begin{center}
        \begin{tabular}{|| c | c c || c c ||}
            \hline
            \multirow{2}{*}{št. zagona} & \multicolumn{2}{c||}{točnost najboljšega agenta} & \multicolumn{2}{c||}{MKK najboljšega agenta} \\ \cline{2-5}
            & učna   & testna          & učna  & testna                  \\
            \hline
            1         & 88.6\% & 89.2\%          & 0.755 & 0.762                   \\
            \hline
            2         & 92.6\% & \textbf{93.0\%} & 0.755 & 0.762                   \\
            \hline
            3         & 90.1\% & 90.6\%          & 0.749 & 0.757                   \\
            \hline
            4         & 88.0\% & 88.6\%          & 0.653 & 0.657                   \\
            \hline
            5         & 90.6\% & 91.0\%          & 0.785 & \textbf{0.790 (93.0\%)} \\
            \hline
            povprečje & 90.0\% & 90.5\%          & 0.739 & 0.746                   \\
            \hline
            $\sigma$  & 0.016  & 0.015           & 0.045 & 0.046                   \\
            \hline
        \end{tabular}
    \end{center}
    \caption{Rezultat tretjega nabora parametrov.}
    \label{tab:statlog_result_3}
\end{table}

%\begin{table}[H]
%    \centering
%    \begin{tabular}{||rcccccccc||}
%        \hline
%        razred    & RF    & FC & FO & High & Bypass & BC & BO & vsota \\ \hline
%        Rad Flow  & 11470 & 0  & 0  & 2    & 0      & 1  & 5  & 11478 \\ \hline
%        Fpv Close & 6     & 0  & 0  & 7    & 0      & 0  & 0  & 13    \\ \hline
%        Fpv Open  & 18    & 0  & 0  & 20   & 0      & 0  & 1  & 39    \\ \hline
%        High      & 947   & 0  & 0  & 1208 & 0      & 0  & 0  & 2155  \\ \hline
%        Bypass    & 0     & 0  & 0  & 0    & 807    & 3  & 2  & 809   \\ \hline
%        Bpv Close & 1     & 0  & 0  & 3    & 0      & 0  & 0  & 4     \\ \hline
%        Bpv Open  & 1     & 0  & 0  & 0    & 0      & 0  & 1  & 2     \\ \hline
%        vsota     & 12443 & 0  & 0  & 1240 & 807    & 1  & 9  & 14500 \\ \hline
%    \end{tabular}
%    \caption{Matrika zmot najbolj točnega agenta tretjega nabora. Agent ne more napovedati razredov \enquote{Fpv Close} in \enquote{Fpv Open}.}
%    \label{tab:statlog_acc_3}
%\end{table}
%
%\begin{table}[H]
%    \centering
%    \begin{tabular}{||rcccccccc||}
%        \hline
%        razred    & RF    & FC & FO & High & Bypass & BC & BO & vsota \\ \hline
%        Rad Flow  & 11456 & 0  & 8  & 9    & 1      & 0  & 4  & 11478 \\ \hline
%        Fpv Close & 6     & 0  & 0  & 4    & 3      & 0  & 0  & 13    \\ \hline
%        Fpv Open  & 18    & 0  & 0  & 21   & 0      & 0  & 0  & 39    \\ \hline
%        High      & 893   & 0  & 0  & 1262 & 0      & 0  & 0  & 2155  \\ \hline
%        Bypass    & 0     & 0  & 1  & 41   & 767    & 0  & 0  & 809   \\ \hline
%        Bpv Close & 0     & 0  & 0  & 0    & 4      & 0  & 0  & 4     \\ \hline
%        Bpv Open  & 0     & 0  & 0  & 0    & 0      & 0  & 2  & 2     \\ \hline
%        vsota     & 12373 & 0  & 9  & 1337 & 775    & 0  & 6  & 14500 \\ \hline
%    \end{tabular}
%    \caption{Matrika zmot agenta z največjim MKK tretjega nabora. Agent ne more napovedati razredov \enquote{Fpv Close} in \enquote{Bpv Close}.}
%    \label{tab:statlog_mcc_3}
%\end{table}

\begin{figure}[H]
    \begin{center}
        \includegraphics[width=13cm]{shuttle/3/acc}
    \end{center}
    \caption{Graf točnosti populacije najboljšega agenta tretjega nabora skozi generacije.}
    \label{fig:statlog_acc_3}
\end{figure}

\begin{figure}[H]
    \begin{center}
        \includegraphics[width=13cm]{shuttle/3/mcc}
    \end{center}
    \caption{Graf MKK populacije najboljšega agenta tretjega nabora skozi generacije.}
    \label{fig:statlog_mcc_3}
\end{figure}

\begin{figure}[H]
    \begin{center}
        \includegraphics[width=13cm]{shuttle/3/acc_g}
    \end{center}
    \caption{Vizualizacija najbolj točnega agenta tretjega nabora. Vsebuje 1 vmesno vozlišče in 22 povezav.}
    \label{fig:statlog_acc_3_g}
\end{figure}

\begin{figure}[H]
    \begin{center}
        \includegraphics[width=13cm]{shuttle/3/mcc_g}
    \end{center}
    \caption{Vizualizacija agenta z največjim MKK tretjega nabora. Vsebuje 2 vmesni vozlišči in 27 povezav.}
    \label{fig:statlog_mcc_3_g}
\end{figure}

\section{Random forest}\label{sec:random-forest-test}
\subsection{Iris}\label{subsec:random-forest-iris-test}
\begin{table}[H]
    \begin{center}
        \begin{tabular}{|| c | c c ||}
            \hline
            število dreves & točnost & MKK   \\
            \hline
            100            & 1.000   & 1.000 \\
            \hline
            150            & 1.000   & 1.000 \\
            \hline
            200            & 1.000   & 1.000 \\
            \hline
        \end{tabular}
    \end{center}
    \caption{Rezultat random forest pristopa na množici Iris.}
    \label{tab:rforest_iris_result}
\end{table}

\subsection{Wine}\label{subsec:random-forest-wine-test}
\begin{table}[H]
    \begin{center}
        \begin{tabular}{|| c | c c ||}
            \hline
            število dreves & točnost & MKK   \\
            \hline
            200            & 0.962   & 0.945 \\
            \hline
            250            & 0.962   & 0.945 \\
            \hline
            350            & 0.962   & 0.945 \\
            \hline
        \end{tabular}
    \end{center}
    \caption{Rezultat random forest pristopa na množici Wine.}
    \label{tab:rforest_wine_result}
\end{table}

\begin{table}[H]
    \centering
    \begin{tabular}{||rcccc||}
        \hline
        razred  & Class 1 & Class 2 & Class 3 & vsota \\ \hline
        Class 1 & 18      & 0       & 0       & 18    \\ \hline
        Class 2 & 0       & 19      & 2       & 21    \\ \hline
        Class 3 & 0       & 0       & 14      & 14    \\ \hline
        vsota   & 18      & 19      & 16      & 53    \\ \hline
    \end{tabular}
    \caption{Matrika zmot random forest pristopa na množici Wine. Napačne so samo napovedi razreda \enquote{Class 2}.
    Matrika zmot je enaka pri vseh treh številih dreves.}
    \label{tab:rforest_wine_cm}
\end{table}

\subsection{Car}\label{subsec:random-forest-car-test}
\begin{table}[H]
    \begin{center}
        \begin{tabular}{|| c | c c ||}
            \hline
            število dreves & točnost & MKK   \\
            \hline
            200            & 0.956   & 0.903 \\
            \hline
            250            & 0.958   & 0.908 \\
            \hline
            300            & 0.959   & 0.912 \\
            \hline
        \end{tabular}
    \end{center}
    \caption{Rezultat random forest pristopa na množici Car.}
    \label{tab:rforest_car_result}
\end{table}

\begin{table}[H]
    \centering
    \begin{tabular}{||rccccc||}
        \hline
        razred       & unacceptable & acceptable & good & very good & vsota \\ \hline
        unacceptable & 359          & 4          & 0    & 0         & 363   \\ \hline
        acceptable   & 1            & 108        & 4    & 2         & 115   \\ \hline
        good         & 0            & 7          & 12   & 2         & 21    \\ \hline
        very good    & 0            & 3          & 0    & 16        & 19    \\ \hline
        vsota        & 360          & 122        & 16   & 20        & 518   \\ \hline
    \end{tabular}
    \caption{Matrika zmot random forest pristopa na množici Wine z 200 drevesi.}
    \label{tab:rforest_car_cm_1}
\end{table}

\begin{table}[H]
    \centering
    \begin{tabular}{||rccccc||}
        \hline
        razred       & unacceptable & acceptable & good & very good & vsota \\ \hline
        unacceptable & 360          & 3          & 0    & 0         & 363   \\ \hline
        acceptable   & 0            & 109        & 4    & 2         & 115   \\ \hline
        good         & 0            & 9          & 10   & 2         & 21    \\ \hline
        very good    & 0            & 2          & 0    & 17        & 19    \\ \hline
        vsota        & 360          & 123        & 14   & 21        & 518   \\ \hline
    \end{tabular}
    \caption{Matrika zmot random forest pristopa na množici Wine z 250 drevesi.}
    \label{tab:rforest_car_cm_2}
\end{table}

\begin{table}[H]
    \centering
    \begin{tabular}{||rccccc||}
        \hline
        razred       & unacceptable & acceptable & good & very good & vsota \\ \hline
        unacceptable & 359          & 4          & 0    & 0         & 363   \\ \hline
        acceptable   & 0            & 109        & 4    & 2         & 115   \\ \hline
        good         & 0            & 5          & 14   & 2         & 21    \\ \hline
        very good    & 0            & 4          & 0    & 15        & 19    \\ \hline
        vsota        & 359          & 122        & 18   & 19        & 518   \\ \hline
    \end{tabular}
    \caption{Matrika zmot random forest pristopa na množici Wine s 300 drevesi.}
    \label{tab:rforest_car_cm_3}
\end{table}

\subsection{Shuttle}\label{subsec:random-forest-shuttle-test}
\begin{table}[H]
    \begin{center}
        \begin{tabular}{|| c | c c ||}
            \hline
            število dreves & točnost & MKK   \\
            \hline
            200            & 1.000   & 0.999 \\
            \hline
            250            & 1.000   & 0.999 \\
            \hline
            300            & 1.000   & 0.999 \\
            \hline
        \end{tabular}
    \end{center}
    \caption{Rezultat random forest pristopa na množici Shuttle.}
    \label{tab:rforest_shuttle_result}
\end{table}

\begin{table}[H]
    \centering
    \begin{tabular}{||rcccccccc||}
        \hline
        razred    & RF    & FC & FO & High & Bypass & BC & BO & vsota \\ \hline
        Rad Flow  & 11477 & 0  & 1  & 0    & 0      & 0  & 0  & 11478 \\ \hline
        Fpv Close & 0     & 12 & 0  & 1    & 0      & 0  & 0  & 13    \\ \hline
        Fpv Open  & 1     & 0  & 38 & 0    & 0      & 0  & 0  & 39    \\ \hline
        High      & 0     & 0  & 0  & 2155 & 0      & 0  & 0  & 2155  \\ \hline
        Bypass    & 1     & 0  & 0  & 0    & 808    & 0  & 0  & 809   \\ \hline
        Bpv Close & 0     & 0  & 0  & 1    & 1      & 2  & 0  & 4     \\ \hline
        Bpv Open  & 0     & 0  & 1  & 0    & 0      & 0  & 1  & 2     \\ \hline
        vsota     & 11479 & 12 & 40 & 2157 & 809    & 2  & 1  & 14500 \\ \hline
    \end{tabular}
    \caption{Matrika zmot random forest pristopa na množici Shuttle z 200 drevesi.}
    \label{tab:rforest_shuttle_cm_1}
\end{table}

\begin{table}[H]
    \centering
    \begin{tabular}{||rcccccccc||}
        \hline
        razred    & RF    & FC & FO & High & Bypass & BC & BO & vsota \\ \hline
        Rad Flow  & 11477 & 0  & 1  & 0    & 0      & 0  & 0  & 11478 \\ \hline
        Fpv Close & 0     & 12 & 0  & 1    & 0      & 0  & 0  & 13    \\ \hline
        Fpv Open  & 1     & 0  & 38 & 0    & 0      & 0  & 0  & 39    \\ \hline
        High      & 0     & 0  & 0  & 2155 & 0      & 0  & 0  & 2155  \\ \hline
        Bypass    & 1     & 0  & 0  & 0    & 808    & 0  & 0  & 809   \\ \hline
        Bpv Close & 1     & 0  & 0  & 0    & 1      & 2  & 0  & 4     \\ \hline
        Bpv Open  & 0     & 0  & 1  & 0    & 0      & 0  & 1  & 2     \\ \hline
        vsota     & 11480 & 12 & 40 & 2156 & 809    & 2  & 1  & 14500 \\ \hline
    \end{tabular}
    \caption{Matrika zmot random forest pristopa na množici Shuttle z 250 drevesi.}
    \label{tab:rforest_shuttle_cm_2}
\end{table}

\begin{table}[H]
    \centering
    \begin{tabular}{||rcccccccc||}
        \hline
        razred    & RF    & FC & FO & High & Bypass & BC & BO & vsota \\ \hline
        Rad Flow  & 11477 & 0  & 1  & 0    & 0      & 0  & 0  & 11478 \\ \hline
        Fpv Close & 0     & 12 & 0  & 1    & 0      & 0  & 0  & 13    \\ \hline
        Fpv Open  & 0     & 0  & 39 & 0    & 0      & 0  & 0  & 39    \\ \hline
        High      & 0     & 0  & 0  & 2155 & 0      & 0  & 0  & 2155  \\ \hline
        Bypass    & 0     & 0  & 0  & 0    & 809    & 0  & 0  & 809   \\ \hline
        Bpv Close & 0     & 0  & 0  & 1    & 1      & 2  & 0  & 4     \\ \hline
        Bpv Open  & 0     & 0  & 1  & 0    & 0      & 0  & 1  & 2     \\ \hline
        vsota     & 11477 & 12 & 41 & 2157 & 810    & 2  & 1  & 14500 \\ \hline
    \end{tabular}
    \caption{Matrika zmot random forest pristopa na množici Shuttle s 300 drevesi.}
    \label{tab:rforest_shuttle_cm_3}
\end{table}

