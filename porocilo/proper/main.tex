%%%%%%%%%%%%%%%%%%%%%%%%%%%%%%%%%%%%%%%%
% datoteka diploma-FRI-vzorec.tex
%
%POZOR: ta verzija ne producira pdf datoteke v pdf/A formatu!!!
%namenjena je le za nalogo pri Diplomskem seminarju!
%
% vzorčna datoteka za pisanje diplomskega dela v formatu LaTeX
% na UL Fakulteti za računalništvo in informatiko
%
% na osnovi starejših verzij vkup spravil Franc Solina, maj 2021
% prvo verzijo je leta 2010 pripravil Gašper Fijavž
%
% za upravljanje z literaturo ta vezija uporablja BibLaTeX
%
% svetujemo uporabo Overleaf.com - na tej spletni implementaciji LaTeXa ta vzorec zagotovo pravilno deluje
%

\documentclass[a4paper,12pt,openright]{book}
%\documentclass[a4paper, 12pt, openright, draft]{book}  Nalogo preverite tudi z opcijo draft, ki pokaže, katere vrstice so predolge! Pozor, v draft opciji, se slike ne pokažejo!

\usepackage[utf8]{inputenc}   % omogoča uporabo slovenskih črk kodiranih v formatu UTF-8
\usepackage[slovene,english]{babel}    % naloži, med drugim, slovenske delilne vzorce
\usepackage[pdftex]{graphicx}  % omogoča vlaganje slik različnih formatov
\usepackage{fancyhdr}          % poskrbi, na primer, za glave strani
\usepackage{amssymb}           % dodatni matematični simboli
\usepackage{amsmath}           % eqref, npr.
\usepackage{hyperxmp}
\usepackage[hyphens]{url}
\usepackage{csquotes}
\usepackage[pdftex, colorlinks=true,
    citecolor=black, filecolor=black,
    linkcolor=black, urlcolor=black,
    pdfproducer={LaTeX}, pdfcreator={LaTeX}]{hyperref}

\usepackage{color}
\usepackage{soul}

\usepackage[
    backend=biber,
    style=numeric,
    sorting=nty,
]{biblatex}


\addbibresource{literatura.bib} %Imports bibliography file


%%%%%%%%%%%%%%%%%%%%%%%%%%%%%%%%%%%%%%%%
%	DIPLOMA INFO
%%%%%%%%%%%%%%%%%%%%%%%%%%%%%%%%%%%%%%%%
\newcommand{\ttitle}{Nevroevolucija za strojno učenje}
\newcommand{\ttitleEn}{Neuroevolution for machine learning}
\newcommand{\tsubject}{\ttitle}
\newcommand{\tsubjectEn}{\ttitleEn}
\newcommand{\tauthor}{Jure Vreček}
\newcommand{\tkeywords}{nevroevolucija, strojno učenje, rekurentne nevronske mreže, genetski algoritem}
\newcommand{\tkeywordsEn}{neuroevolution, machine learning, recurrent neural networks, genetic algorithm}

%%%%%%%%%%%%%%%%%%%%%%%%%%%%%%%%%%%%%%%%
%	HYPERREF SETUP
%%%%%%%%%%%%%%%%%%%%%%%%%%%%%%%%%%%%%%%%
\hypersetup{pdftitle={\ttitle}}
\hypersetup{pdfsubject=\ttitleEn}
\hypersetup{pdfauthor={\tauthor}}
\hypersetup{pdfkeywords=\tkeywordsEn}

%%%%%%%%%%%%%%%%%%%%%%%%%%%%%%%%%%%%%%%%
% postavitev strani
%%%%%%%%%%%%%%%%%%%%%%%%%%%%%%%%%%%%%%%%

\addtolength{\marginparwidth}{-20pt} % robovi za tisk
\addtolength{\oddsidemargin}{40pt}
\addtolength{\evensidemargin}{-40pt}

\renewcommand{\baselinestretch}{1.3} % ustrezen razmik med vrsticami
\setlength{\headheight}{15pt}        % potreben prostor na vrhu
\renewcommand{\chaptermark}[1]%
{\markboth{\MakeUppercase{\thechapter.\ #1}}{}} \renewcommand{\sectionmark}[1]%
{\markright{\MakeUppercase{\thesection.\ #1}}} \renewcommand{\headrulewidth}{0.5pt} \renewcommand{\footrulewidth}{0pt}
\fancyhf{}
\fancyhead[LE,RO]{\sl \thepage}
%\fancyhead[LO]{\sl \rightmark} \fancyhead[RE]{\sl \leftmark}
\fancyhead[RE]{\sc \tauthor}              % dodal Solina
\fancyhead[LO]{\sc Diplomska naloga}     % dodal Solina


\newcommand{\BibLaTeX}{{\sc Bib}\LaTeX}
\newcommand{\BibTeX}{{\sc Bib}\TeX}

%%%%%%%%%%%%%%%%%%%%%%%%%%%%%%%%%%%%%%%%
% naslovi
%%%%%%%%%%%%%%%%%%%%%%%%%%%%%%%%%%%%%%%%

\newcommand{\autfont}{\Large}
\newcommand{\titfont}{\LARGE\bf}
\newcommand{\clearemptydoublepage}{\newpage{\pagestyle{empty}\cleardoublepage}}
\setcounter{tocdepth}{1}          % globina kazala

%%%%%%%%%%%%%%%%%%%%%%%%%%%%%%%%%%%%%%%%
% konstrukti
%%%%%%%%%%%%%%%%%%%%%%%%%%%%%%%%%%%%%%%%
\newtheorem{izrek}{Izrek}[chapter]
\newtheorem{trditev}{Trditev}[izrek]
\newenvironment{dokaz}{\emph{Dokaz.}\ }{\hspace{\fill}{$\Box$}}


%%%%%%%%%%%%%%%%%%%%%%%%%%%%%%%%%%%%%%%%%%%%%%%%%%%%%%%%%%%%%%%%%%%%%%%%%%%%%%%
%% PDF-A
%%%%%%%%%%%%%%%%%%%%%%%%%%%%%%%%%%%%%%%%%%%%%%%%%%%%%%%%%%%%%%%%%%%%%%%%%%%%%%%

%%%%%%%%%%%%%%%%%%%%%%%%%%%%%%%%%%%%%%%%
% define medatata
%%%%%%%%%%%%%%%%%%%%%%%%%%%%%%%%%%%%%%%%
\def\Title{\ttitle}
\def\Author{\tauthor, jv9074@fri.uni-lj.si}
\def\Subject{\ttitleEn}
\def\Keywords{\tkeywordsEn}

%%%%%%%%%%%%%%%%%%%%%%%%%%%%%%%%%%%%%%%%
% \convertDate converts D:20080419103507+02'00' to 2008-04-19T10:35:07+02:00
%%%%%%%%%%%%%%%%%%%%%%%%%%%%%%%%%%%%%%%%
\def\convertDate{%
    \getYear
}

{\catcode`\D=12
\gdef\getYear D:#1#2#3#4{\edef\xYear{#1#2#3#4}\getMonth}
}
\def\getMonth#1#2{\edef\xMonth{#1#2}\getDay}
\def\getDay#1#2{\edef\xDay{#1#2}\getHour}
\def\getHour#1#2{\edef\xHour{#1#2}\getMin}
\def\getMin#1#2{\edef\xMin{#1#2}\getSec}
\def\getSec#1#2{\edef\xSec{#1#2}\getTZh}
\def\getTZh +#1#2{\edef\xTZh{#1#2}\getTZm}
\def\getTZm '#1#2'{%
    \edef\xTZm{#1#2}%
    \edef\convDate{\xYear-\xMonth-\xDay T\xHour:\xMin:\xSec+\xTZh:\xTZm}%
}

%\expandafter\convertDate\pdfcreationdate

%%%%%%%%%%%%%%%%%%%%%%%%%%%%%%%%%%%%%%%%
% get pdftex version string
%%%%%%%%%%%%%%%%%%%%%%%%%%%%%%%%%%%%%%%%
\newcount\countA
\countA=\pdftexversion
\advance \countA by -100
\def\pdftexVersionStr{pdfTeX-1.\the\countA.\pdftexrevision}
\interfootnotelinepenalty=10000

%%%%%%%%%%%%%%%%%%%%%%%%%%%%%%%%%%%%%%%%
% XMP data
%%%%%%%%%%%%%%%%%%%%%%%%%%%%%%%%%%%%%%%%
\usepackage{xmpincl}
%\includexmp{pdfa-1b}

%%%%%%%%%%%%%%%%%%%%%%%%%%%%%%%%%%%%%%%%
% pdfInfo
%%%%%%%%%%%%%%%%%%%%%%%%%%%%%%%%%%%%%%%%
\pdfinfo{%
    /Title    (\ttitle)
    /Author   (\tauthor, damjan@cvetan.si)
    /Subject  (\ttitleEn)
    /Keywords (\tkeywordsEn)
    /ModDate  (\pdfcreationdate)
    /Trapped /False
}

%%%%%%%%%%%%%%%%%%%%%%%%%%%%%%%%%%%%%%%%
% znaki za copyright stran
%%%%%%%%%%%%%%%%%%%%%%%%%%%%%%%%%%%%%%%%

\newcommand{\CcImageCc}[1]{%
    \includegraphics[scale=#1]{cc_cc_30.pdf}%
}
\newcommand{\CcImageBy}[1]{%
    \includegraphics[scale=#1]{cc_by_30.pdf}%
}
\newcommand{\CcImageSa}[1]{%
    \includegraphics[scale=#1]{cc_sa_30.pdf}%
}

%%%%%%%%%%%%%%%%%%%%%%%%%%%%%%%%%%%%%%%%%%%%%%%%%%%%%%%%%%%%%%%%%%%%%%%%%%%%%%%
%%%%%%%%%%%%%%%%%%%%%%%%%%%%%%%%%%%%%%%%%%%%%%%%%%%%%%%%%%%%%%%%%%%%%%%%%%%%%%%

\begin{document}
    \selectlanguage{slovene}
    \frontmatter
    \setcounter{page}{1} %
    \renewcommand{\thepage}{}       % preprečimo težave s številkami strani v kazalu

%%%%%%%%%%%%%%%%%%%%%%%%%%%%%%%%%%%%%%%%
%naslovnica

    \thispagestyle{empty}%
    \begin{center}
    {\large\sc Univerza v Ljubljani\\%
    Fakulteta za računalništvo in informatiko\\%
    }
        \vskip 10em%
            {\autfont \tauthor\par}%
            {\titfont \ttitle \par}%
            {\vskip 3em \textsc{DIPLOMSKO DELO\\[5mm]         % dodal Solina za ostale študijske programe
%    VISOKOŠOLSKI STROKOVNI ŠTUDIJSKI PROGRAM\\ PRVE STOPNJE\\ RAČUNALNIŠTVO IN INFORMATIKA}\par}%
        UNIVERZITETNI ŠTUDIJSKI PROGRAM\\ PRVE STOPNJE\\ RAČUNALNIŠTVO IN INFORMATIKA}\par}%
%    INTERDISCIPLINARNI UNIVERZITETNI\\ ŠTUDIJSKI PROGRAM PRVE STOPNJE\\ MULTIMEDIJA}\par}%
%    INTERDISCIPLINARNI UNIVERZITETNI\\ ŠTUDIJSKI PROGRAM PRVE STOPNJE\\ UPRAVNA INFORMATIKA}\par}%
%    INTERDISCIPLINARNI UNIVERZITETNI\\ ŠTUDIJSKI PROGRAM PRVE STOPNJE\\ RAČUNALNIŠTVO IN MATEMATIKA}\par}%
        \vfill\null%
% izberite pravi habilitacijski naziv mentorja!
        {\large \textsc{Mentor}: prof. dr. Marko Robnik Šikonja\par}%
%   {\large \textsc{Somentor}:  viš. pred./doc./izr. prof./prof. dr.  Martin Krpan \par}%
        {\vskip 2em \large Ljubljana, \the\year \par}%
    \end{center}
% prazna stran
%\clearemptydoublepage
% izjava o licencah itd. se izpiše na hrbtni strani naslovnice

%%%%%%%%%%%%%%%%%%%%%%%%%%%%%%%%%%%%%%%%
%copyright stran
%%%%%%%%%%%%%%%%%%%%%%%%%%%%%%%%%%%%%%%%
    \newpage
    \thispagestyle{empty}

    \vspace*{5cm}
    {\small \noindent
    To delo je ponujeno pod licenco \textit{Creative Commons Priznanje avtorstva-Deljenje pod enakimi pogoji 2.5 Slovenija} (ali novej\v so razli\v cico).
    To pomeni, da se tako besedilo, slike, grafi in druge sestavine dela kot tudi rezultati diplomskega dela lahko prosto distribuirajo,
        reproducirajo, uporabljajo, priobčujejo javnosti in predelujejo, pod pogojem, da se jasno in vidno navede avtorja in naslov tega
    dela in da se v primeru spremembe, preoblikovanja ali uporabe tega dela v svojem delu, lahko distribuira predelava le pod
    licenco, ki je enaka tej.
    Podrobnosti licence so dostopne na spletni strani \href{http://creativecommons.si}{creativecommons.si} ali na Inštitutu za
    intelektualno lastnino, Streliška 1, 1000 Ljubljana.

    \vspace*{1cm}
        \begin{center}% 0.66 / 0.89 = 0.741573033707865
            \CcImageCc{0.741573033707865}\hspace*{1ex}\CcImageBy{1}\hspace*{1ex}\CcImageSa{1}%
        \end{center}
    }

    \vspace*{1cm}
    {\small \noindent
    Izvorna koda diplomskega dela, njeni rezultati in v ta namen razvita programska oprema je ponujena pod licenco GNU General Public License,
        različica 3 (ali novejša). To pomeni, da se lahko prosto distribuira in/ali predeluje pod njenimi pogoji.
    Podrobnosti licence so dostopne na spletni strani \url{http://www.gnu.org/licenses/}.
    }

    \vfill
    \begin{center}
        \ \\ \vfill
        {\em
        Besedilo je oblikovano z urejevalnikom besedil \LaTeX.}
    \end{center}

% prazna stran
    \clearemptydoublepage

%%%%%%%%%%%%%%%%%%%%%%%%%%%%%%%%%%%%%%%%
% stran 3 med uvodnimi listi
    \thispagestyle{empty}
    \
    \vfill

    \bigskip
    \noindent\textbf{Kandidat:} Jure Vreček\\
    \noindent\textbf{Naslov:} Neuroevolucija za strojno učenje\\
% vstavite ustrezen naziv študijskega programa!
    \noindent\textbf{Vrsta naloge:} Diplomska naloga na univerzitetnem programu prve stopnje Računalništvo in informatika \\
% izberite pravi habilitacijski naziv mentorja!
    \noindent\textbf{Mentor:} prof. dr. Marko Robnik Šikonja\\
    %\noindent\textbf{Somentor:} isto kot za mentorja

    \bigskip
    \noindent\textbf{Opis:}\\
    Besedilo teme diplomskega dela študent prepiše iz študijskega informacijskega sistema, kamor ga je vnesel mentor.
    V nekaj stavkih bo opisal, kaj pričakuje od kandidatovega diplomskega dela.
    Kaj so cilji, kakšne metode naj uporabi, morda bo zapisal tudi ključno literaturo.

    \bigskip
    \noindent\textbf{Title:} Neuroevolution for machine learning

    \bigskip
    \noindent\textbf{Description:}\\
    opis diplome v angleščini

    \vfill



    \vspace{2cm}

% prazna stran
    \clearemptydoublepage

% zahvala
    \thispagestyle{empty}\mbox{}\vfill\null\it%
    \noindent
    Zahvaljujem se svojemu mentorju za njegovo potrpežljivost.
    \rm\normalfont

% prazna stran
    \clearemptydoublepage

%%%%%%%%%%%%%%%%%%%%%%%%%%%%%%%%%%%%%%%%
% posvetilo, če sama zahvala ne zadošča :-)
    %\thispagestyle{empty}\mbox{}{\vskip0.20\textheight}\mbox{}\hfill\begin{minipage}{0.55\textwidth}%
    %                                                                    Svoji dragi Alenčici.
    %                                                                    \normalfont
    %\end{minipage}

% prazna stran
    \clearemptydoublepage


%%%%%%%%%%%%%%%%%%%%%%%%%%%%%%%%%%%%%%%%
% kazalo
    \pagestyle{empty}
    \def\thepage{}% preprečimo težave s številkami strani v kazalu
    \tableofcontents{}


% prazna stran
    \clearemptydoublepage

%%%%%%%%%%%%%%%%%%%%%%%%%%%%%%%%%%%%%%%%
% seznam kratic

    \chapter*{Seznam uporabljenih kratic}

    \noindent\begin{tabular}{p{0.11\textwidth}|p{.39\textwidth}|p{.39\textwidth}}    % po potrebi razširi prvo kolono tabele na račun drugih dveh!
    \textbf{kratica}
                 & \textbf{angleško}             & \textbf{slovensko}                        \\ \hline
                 \textbf{CA}    &   classification accuracy &   klasifikacijska točnost \\
                 \textbf{ANN}   &   artificial neural network   &   umetna nevronska mreža \\
                 \textbf{NN}    &   neural network    & nevronska mreža \\
                 \textbf{ReLU}    &   rectified linear unit    & TBD \\
%  \dots & \dots & \dots \\
    \end{tabular}



    \clearemptydoublepage

%%%%%%%%%%%%%%%%%%%%%%%%%%%%%%%%%%%%%%%%
% povzetek
    \addcontentsline{toc}{chapter}{Povzetek}
    \chapter*{Povzetek}

    \noindent\textbf{Naslov:} \ttitle
    \bigskip

    \noindent\textbf{Avtor:} \tauthor
    \bigskip

%\noindent\textbf{Povzetek:}
    %\noindent V vzorcu je predstavljen postopek priprave diplomskega dela z uporabo okolja \LaTeX. Vaš povzetek mora sicer vsebovati približno 100 besed, ta tukaj je odločno prekratek.
    %Dober povzetek vključuje: (1) kratek opis obravnavanega problema, (2) kratek opis vašega pristopa za reševanje tega problema in (3) (najbolj uspešen) rezultat ali prispevek diplomske naloge.
    \noindent To diplomsko delo pokriva opis izdelave rekurentnih nevronskih mrež s pomočjo namenskega programa, ki
    pokriva inicializacijo začetne populacije, vzorčenje, križanje, mutiranje in izračun kvalitete.
    Nevronske mreže, ki jih program izdela, so namenjene večrazredni klasifikaciji nad podatkovnimi množicami z zveznimi ali distkretnimi atributi.
    Algoritmi inicializacije, križanja in mutiranja so specifični za namen tega dela.
    Vzorčenje je implementirano s prilagojenim stohastičnim univerzalnem vzorčenjem, kjer se vedno obdrži najboljši agent prejšnje
    generacije, kvaliteta pa se izračuna na podlagi točnosti napovedi ali pa z večrazrednim Matthewsovim korelacijskim koeficientom s pomočjo matrike zmot.
    Pri obeh metodah se lahko upošteva tudi število vozlišč in povezav posamezne nevronske mreže.
    Program je bil razvit s pomočjo množic Iris, Wine, Blood Transfusion Service Center, Car Evaluation in Statlog (Shuttle), ki so
    na voljo na UCI Machine Learning Repository~\cite{Dua:2019}.
    V tem poročilu je vključena tudi primerjava z nekaterimi odprtokodnimi rešitvami strojnega učenja, ki so trenutno na voljo.

    \bigskip

    \noindent\textbf{Ključne besede:} \tkeywords.
% prazna stran
    \clearemptydoublepage

%%%%%%%%%%%%%%%%%%%%%%%%%%%%%%%%%%%%%%%%
% abstract
    \selectlanguage{english}
    \addcontentsline{toc}{chapter}{Abstract}
    \chapter*{Abstract}

    \noindent\textbf{Title:} \ttitleEn
    \bigskip

    \noindent\textbf{Author:} \tauthor
    \bigskip

%\noindent\textbf{Abstract:}
    \noindent This diploma thesis describes the creation of recurrent neural networks with a custom program that covers the
    initialization of the initial population, sampling, crossover, mutation and fitness calculation.
    Neural networks, created by the program, are designed for multiclass classification of data in data sets with continuous or discrete attributes.
    Initialization, crossover and mutation algorithms are specific for this thesis.
    Crossover is implemented with a modified stochastic universal sampling algorithm, where the previous generation's fittest agent is always kept.
    Fitness is calculated based on prediction accuracy or on the Matthews correlation coefficient with the help of a confusion matrix.
    Both methods can also take into account the number of vertices and edges of each neural network.
    The program was developed with the help of the following datasets, which are available in the UCI Machine Learning Repository~\cite{Dua:2019}:
    Iris, Wine, Blood Transfusion Service Center, Car Evaluation and Statlog (Shuttle).
    This report also includes comparisons with a few open source machine learning solutions available at the time of writing.
    \bigskip

    \noindent\textbf{Keywords:} \tkeywordsEn.
    \selectlanguage{slovene}
% prazna stran
    \clearemptydoublepage

%%%%%%%%%%%%%%%%%%%%%%%%%%%%%%%%%%%%%%%%
    \mainmatter
    \setcounter{page}{1}
    \pagestyle{fancy}


    \chapter{Uvod}\label{ch:uvod}
    Ob pisanju tega poročila je na področju strojnega učenja v akademskih in komercialnih sferah izjemno priljubljena
    uporaba nevronskih mrež za napovedovanje rezultatov na podlagi vnaprej pripravljenih podatkovnih množic in za reševanje
    problemov s principi spodbujevalnega učenja.
    Topologija teh nevronskih mrež se giblje od najbolj enostavnih brez skritih nivojev, do izjemno kompleksnih večnivojskih
    z dvosmernimi povezavami.
    Učenje le-teh pa velikokrat poteka z metodo vzvratnega razširjanja, kjer se prilagodi le utež povezav med vozlišči,
    topologija pa ostane enaka.
    Motivacija za tem diplomskim delom je raziskati, ali učenje z metodo nevroevolucije, kjer naša implementacija poleg uteži nevronskih
    mrež spreminja tudi topologijo, prinese dodano vrednost pri točnosti, hitrosti učenja in velikosti končnih nevronskih
    mrež v primerjavi z zgoraj omenjenimi rešitvami.

    Program, ki smo ga napisali za raziskovanje tega problema v jeziku C++, ustvarja nevronske mreže na podlagi vhodne
    podatkovne množice, izračuna njihovo kvaliteto, nato pa izbere najboljših $n$ za na naslednjo generacijo (oziroma iteracijo).
    Ta postopek ponavlja, dokler ne zaključi $m$-te iteracije, nato pa v datoteko shrani najboljšo nevronsko mrežo v JSON obliki.
    Več podrobnosti je o programu je na voljo v poglavju~\ref{ch:evolucija-nevronske-mreze}.

    V obseg dela te diplomske naloge poleg programa spada tudi primerjava rezultatov z nekaterimi odprtokodnimi rešitvami (poglavje~\ref{ch:evalvacija})
    in seveda pisanje tega poročila.

    \section{Kaj so nevronske mreže}\label{sec:kaj-so-nevronske-mreze}
    Nevronske mreže (kratica ANN ali NN) so množice povezanih vozlišč oziroma ``nevronov''.
    Njihove lastnosti določajo topologija in lastnosti samih vozlišč~\cite{russell_norvig_2016}.
    Povezave vsebujejo zvezno vrednost uteži, vozlišča pa poljubno aktivacijsko funkcijo, ki transformira vsoto uteženih
    aktivacijskih vrednosti vhodnih vozlišč.
    Preprost model vozlišča je naslednji~\cite{russell_norvig_2016}:
    \begin{equation}
        a_j=f(\sum_{i=0}^{n} a_i w_{i,j})
        \label{eq:neuron_model_splosno}
    \end{equation}
    kjer je:
    \begin{itemize}
        \item $a_j$ aktivacijska vrednost vozlišča
        \item $f(x)$ aktivacijska funkcija
        \item $n$ število vhodnih povezav trenutnega vozlišča
        \item $a_i$ aktivacijska vrednost $i$-tega vhodnega vozlišča
        \item $w_{i,j}$ utež povezave med $i$-tim in trenutnim vozliščem.
    \end{itemize}

    V splošnem vsebujejo fiksno število vhodnih in izhodnih vozlišč ter poljubno število vmesnih oz.\ globokih vozlišč.
    Odvisno od tipa nevronske mreže, so lahko globoka vozlišča združena v skupine, ti.\ sloje.

    \subsection{Rekurentne nevronske mreže}\label{subsec:rekurentne-nevronske-mreze}
    Če zahtevamo, da povezave med vozlišči lahko tvorijo cikel, potem potrebujemo poseben razred nevronskih mrež~\cite{recurrent_neural_network_wiki}.
    Z razliko od acikličnih feed-forward mrež, rekurentne mreže vsebujejo povezave, ki aktivacijsko vrednost nekaterih vozlišč
    sklenejo z njihovimi predhodniki.

    Koncept združevanja globokih vozlišč v sloje je v teh mrežah malce zamegljen.
    Namreč, sloji se pojavijo pri vsakem časovnem koraku propagacije vhodnih vrednosti skozi topologijo.
    Zaradi ciklov je povsem možno, da se neko vozlišče pojavi v več slojih.


    \section{Kaj so genetski algoritmi}\label{sec:kaj-so-genetski-algoritmi}
    Genetski algoritmi črpajo navdih iz procesa naravne selekcije.
    Večinoma so uporabljeni so za reševanje optimizacijskih in iskalnih problemov, ne zagotavljajo pa optimalnih rešitev.
    Namesto enega pravilnega rezultata po navadi najdejo nabor rešitev visoke kvalitete, ki so primerne za nadaljnjo uporabo
    ~\cite{inteligentni_sistemi_2010,genetic_algorithm_wiki_2022}.
    
    V splošnem so sestavljeni iz inicializacije začetne populacije, vzorčenja, križanja, mutiranja in izračuna kvalitete agentov.
    %inicializacijo začetne populacije, vzorčenje, križanje, mutiranje in izračun kvalitete
    \begin{description}
        \item[Agent]{je predstavitev posamezne vmesne ali končne rešitve. Lahko je zgrajen v obliki vektorja binarnih ali realnih
            števil, znakov, ali pa v obliki drevesa oz. grafa. Posamezno vrednost imenujemo gen, ki lahko predstavlja navodilo za
            reševanje problema (npr. premik gor, dol, levo ali desno pri sprehodu skozi labirint), lahko pa predstavlja
            vrednost nekega optimiziranega parametra. }
        \item[Izračun kvalitete]{agentov je specifičen glede problem, ki ga želimo rešiti. Vsebino vsakega agenta lahko uporabimo
            kot vhodne parametre našega problema, kvaliteto rezultata pa ustrezno ocenimo. Na primer, pri nevroevoluciji
            je kvaliteta agenta lahko kar točnost klasificiranja.}
        \item[Inicializacija populacije]{je prvi korak genetskega algoritma.
            Tukaj se ustvari začetni nabor naključnih agentov v obliki izbrane predstavitve.
            Velikost populacije je poljubna vrednost, priporočljiva pa je proporcionalnost kompleksnosti agentov in zahtevnosti problema.}
        \item[Vzorčenje]{je proces izbire podmnožice agentov za kreiranje naslednje populacije.
            Največkrat izberemo agente glede na njihovo kvaliteto, zagotoviti pa moramo tudi zadostno raznolikost genetskega
            materjala, da ne obtičimo v lokalnih maksimumih. Poznamo več načinov vzorčenja, na primer~\cite{inteligentni_sistemi_2010}:
            \begin{itemize}
                \item proporcionalna izbira, kjer je verjetnost izbire agenta odvisna od njegove kvalitete glede na populacijo
                \item rangovna izbira, kjer agente sortiramo po kvaliteti padajoče in jim priredimo rang glede na pozicijo, verjetnost izbire pa je odvisna od velikosti ranga
                \item turnirska izbira
                \item stohastično univerzalno vzorčenje, ki je podrobneje opisan v poglavju~\ref{sec:vzorcenje}
            \end{itemize}
            Nič nam ne preprečuje, da nekega agenta izberemo večkrat.
            Taki agenti imajo po navadi v primerjavi s preostalo populacijo zelo visoko kvaliteto, kar pomeni večkratno udeležbo
            v križanju in s tem boljše širjenje kvalitetnih genov v potomce.
            Prav tako ni nujno, da zamenjamo vse agente. V naslednjo populacijo lahko prenesemo nek delež osebkov, pri
            čemer moramo še bolj paziti na raznolikost genetskega materiala. Ta koncept imenujemo elitizem~\cite{inteligentni_sistemi_2010}.
            \item}
        \item[Križanje]{je postopek izmenjave genov med agenti in je zelo odvisen od njihove oblike in tega, kakšno informacijo posamezen gen predstavlja. Na primer, potomce agentov,
            ki so predstavljeni v vektorskem načinu, lahko ustvarimo tako, da vzamemo prvo polovico genov prvega in drugo polovico genov
            drugega starša. Križanje agentov, ki so predstavljeni z drevesnimi strukturami ali grafi, je bistveno težje
            (naša implementacija je opisana v poglavju~\ref{sec:krizanje}).
            Rezultat križanja je lahko en agent ali več, nastopata pa lahko tudi več kot dva starša. }
        \item[Mutiranje]{je postopek naključne spremembe gena/genov agenta. Omogoča obstoj genov, ki jih
            s postopkom križanja ne bi mogli ustvariti. Paziti moramo, število mutiranih osebkov ni preveliko, saj
            smo tako efektivno implementirali naključno preiskovanje rešitev našega problema.}
    \end{description}

    Splošna psevdokoda genetskih algoritmov je naslednja~\cite{inteligentni_sistemi_2010}:
    \begin{verbatim}
var populacija = inicializacija_populacije();
for (dokler nismo zavodoljni z agenti populacije) {
    izračun_kvalitete(populacija);
    vzorčenje(populacija);
    križanje(populacija);
    mutiranje(populacija);
}
    \end{verbatim}


    \section{Kaj je nevroevolucija}\label{sec:kaj-je-nevroevolucija}
    Nevroevolucija je tip strojnega učenja, kjer agente ustvarimo z uporabo genetskih algoritmov.
    Agenti vedno predstavljajo nek tip nevronskih mrež.
    Posamezen gen lahko predstavlja vozlišče ali povezavo (direktno enkodiranje), lahko pa
    predstavlja tudi načrt konstruiranja delčka nevronske mreže (posredno enkodiranje)~\cite{kassahun2007common}.

    Nevroevolucija je zelo primerna na področju spodbujevalnega učenja, kjer nas za izračun kvalitete zanima le uspešnost
    kreirane mreže.
    Na primer, rezultat neke igre lahko enostavno izmerimo (ali je agent zmagal, koliko točk je dobil, kako hiter je bil\ldots),
    vrednost pa uporabimo kot kvaliteto.
    Pri nadzorovanem učenju bi potrebovali podatkovno množico agentovih potez s pričakovanimi rezultati igre.
    V kolikor sama igra vsebuje elemente naključnosti, se izdelava take množice lahko izkaže kot izjemno težaven problem.

    Z razliko od konvencionalnih metod strojnega učenja z vzvratnim razširjanjem, ki operirajo nad nevronskimi mrežami
    fiksne topologije, nevroevolucija lahko optimizira topologijo, uteži povezav med vozlišči in druge parametre naše
    predstavitve mrež.
    Za primer vzemimo parametrični ReLU~\cite{he2015delving}, ki je definiran z:
    \begin{equation}
        f(x)=
        \begin{cases}
            \text{x} & \quad\text{če je x}\ge0\\
            \text{ax} & \quad\text{sicer}\\
        \end{cases}
        \label{eq:parametric_relu}
    \end{equation}
    Če bi se odločili za to aktivacijsko funkcijo, bi v postopku križanja in mutiranja mrež lahko spreminjali tudi
    parameter $a$, ki določa prepustnost negativnih aktivacijskih vrednosti.

    \chapter{Evolucija nevronske mreže}\label{ch:evolucija-nevronske-mreze}
    To poglavje pokriva konkretno implementacijo konceptov v našem programu, opisanih v poglavju~\ref{ch:uvod}.
    Razvoj je potekal s pomočjo podatkovnih množic Iris, Wine, Blood Transfusion Service Center, Car Evaluation in Statlog (Shuttle), ki so
    na voljo na UCI Machine Learning Repository~\cite{Dua:2019}.
    Izdelani program lahko teče nad poljubno množico z zveznimi ali distkretnimi atributi, katere cilj je večrazredna klasifikacija.

    \section{Agenti}\label{sec:agenti}
    Agenti v našem programu so direktno enkodirane rekurentne nevronske mreže v obliki usmerjenega grafa, kjer vsak gen
    predstavlja povezavo ali vozlišče.
    Njihova struktura je delno pogojena z izbrano podatkovno množico;
    število vhodnih vozlišč mora biti enako številu atributov, število izhodnih vozlišč pa mora biti enako številu razredov.
    Količina globokih vozlišč se giblje od 0 do izbrane vrednosti vhodnega parametra.
    Za zgoraj omenjene množice vrednost $20$ predstavlja dovolj dobro razmerje med dovoljeno kompleksnostjo agentov in
    računsko zahtevnostjo.

    V pomnilniku so vozlišča in povezave samostojni objekti, ki hranijo medsebojne reference.
    Matrika sosednosti prostorsko ni najbolj primerna za naše agente, saj bi bila zaradi načina mutiranja (\ref{sec:mutiranje})
    v veliki večini primerov raztresena\footnote{Redke oz. raztresene matrike so matrike, v katerih ima večina elementov vrednost 0.
    V primeru predstavitve nevronskih mrež s tako podatkovno strukturo, vrednost posameznega elementa pomeni utež povezave
    med vozliščema $x_i$ in $y_j$.
    Če je vrednost 0, potem povezava med vozliščema ne obstaja.}.

    V naši implementaciji so vrednosti uteži na nivoju ene nevronske mreže normalizirane med -1 in 1, aktivacijska
    funkcija pa je ``leaky ReLU''~\cite{maas2013rectifier}:
    \begin{equation}
        f(x)=
        \begin{cases}
            \text{x} & \quad\text{če je x}\ge0\\
            \text{0.01x} & \quad\text{sicer}\\
        \end{cases}
        \label{eq:leaky_relu}
    \end{equation}
    Leaky ReLU je variacija ReLU aktivacijske funkcije $f(x)=max(0,x)$, ki dopušča propagiranje negativnih aktivacijskih
    vrednosti skozi nevronsko mrežo.
    Sorodna je parametričnem ReLU iz poglavja~\ref{sec:kaj-je-nevroevolucija}.


    \section{Populacija}

    \section{Inicializacija populacije}

    %inicializacijo začetne populacije, vzorčenje, križanje, mutiranje in izračun kvalitete.
    \section{Vzorčenje}\label{sec:vzorcenje}


    \section{Križanje}\label{sec:krizanje}


    \section{Mutiranje}\label{sec:mutiranje}


    \section{Izračun kvalitete}


    \chapter{Predstavitev metodologije}

    \chapter{Evalvacija}\label{ch:evalvacija}


    \chapter{Zaključki, sklep}



%\cleardoublepage
%\addcontentsline{toc}{chapter}{Literatura}

    \printbibliography[heading=bibintoc,type=article,title={Članki v revijah}]

    \printbibliography[heading=bibintoc,type=inproceedings,title={Članki v zbornikih}]

    \printbibliography[heading=bibintoc,type=incollection,title={Poglavja v knjigah}]

    \printbibliography[heading=bibintoc,title={Celotna literatura}]


\end{document}

